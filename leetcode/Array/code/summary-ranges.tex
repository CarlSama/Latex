    
\begin{description}
    \item{\textbf{Question}}:\\%textbf加粗
		Given a sorted integer array without duplicates, return the summary of its ranges.\\

    \item{\textbf{Example}}\\
		Input : [0,1,2,4,5,7]\\
		Output: ["0->2","4->5","7"]\\

    \item{\textbf{Anslysis}}\\
		因为待处理的数字已经排序状态,所以在融合的时候,可以仅查看已排序状态的结尾\\
		可以借助于stack,每次查看结尾的状态,看能不能融合\\

    \item{\textbf{Solution}}\\
	\item{1. Hash} : \fbox{时间复杂度O(n) , 空间复杂度O(n) }\\
		\begin{lstlisting}
class Solution {
public:
	string toString(int i) {
			string str;
			if(i == 0) {
					str += ('0');
					return str;
			}

			bool isNeg = i < 0;
			long il = ((long)i);
			if(il < 0)
					il *= -1;
			while(il) {//无法处理为0的情况
					str += (il%10 + '0');
					il /= 10;
			}
			if(isNeg)
					str += '-';
			reverse(str.begin(),str.end());
			return str;
	}

    vector<string> summaryRanges(vector<int>& nums) {
	    vector<string>  result;

	    if(nums.size() == 0)	return result;

	    //用于清空之前未保存的值
	    //num.back()+2本身不会被处理
	    nums.push_back(nums.back()-2);

	    int rangeBegin = nums[0],curr;//起码会有1+1个元素
	    for(int idx=1;idx<nums.size();++idx) {
		    curr = nums[idx];

		    if(curr != nums[idx-1] + 1) {
			    string str;
			    if(nums[idx-1] == rangeBegin) {
				    str += toString(rangeBegin);
			    }else {
				    str += toString(rangeBegin);
				    str += "->";
				    str += toString(nums[idx-1]);
			    }
			    result.push_back(str);

			    rangeBegin = curr;
		    }
	    }
	    return result;
    }
};
		\end{lstlisting}

\end{description}

