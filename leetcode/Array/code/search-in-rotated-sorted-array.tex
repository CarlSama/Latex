\begin{description}
    \item{\textbf{Question}}:\\%textbf加粗
		Suppose a sorted array is rotated at some pivot unknown to you beforehand.\\
		(i.e., 0 1 2 4 5 6 7 might become 4 5 6 7 0 1 2).\\
		You are given a target value to search. If found in the array return its index, otherwise return -1.\\
		You may assume no duplicate exists in the array.\\

    \item{\textbf{Anslysis}}\\
		使用循环不变式\\
		若left < right, 则left为所求\\
		若left > right,则求【mid,right】\\
		若left == right,只能一个个找了\\

    \item{\textbf{Solution}}\\
	\item{} : \fbox{时间复杂度O(n) , 空间复杂度O(1) }\\
		\begin{lstlisting}
class Solution {
	int len;
	public:
		int search(vector<int>& nums, int target) {
			len = nums.size();
			int left=0,right=len-1;
			while(left <= right){
				int mid = left + (right - left) /2;
				if(nums[mid] == target)	return true;
				if(nums[mid]==nums[left]&&nums[mid]==nums[right]){
					++left;		--right;
				}else if(nums[mid] >= nums[left]){
					if(target >= nums[left] && target < nums[mid])
						right = mid - 1;
					else 
						left = mid + 1;
				}else{
					if(target <= nums[right] && target > nums[mid])
						left = mid + 1;
					else 
						right = mid - 1;
				}
			}
			return false;
		}

};
		\end{lstlisting}
\end{description}

