    
\begin{description}
    \item{\textbf{Question}}:\\%textbf加粗
		Given a string S and a string T, find the minimum window in S which will contain all the characters in T in complexity O(n).

    \item{\textbf{Example}}\\
		S = "ADOBECODEBANC"\\
		T = "ABC"\\
		Minimum window is "BANC".

    \item{\textbf{Anslysis}}\\
		需要留意在计算某个状态时,可以依据它的前一个状态来更新。\\
		对于T中的每个元素,我们既需要保存它的数目,有需要它的存在性。\\
		不同于连续子串的哈希,这里无法转换成数字来节省空间。\\
		因为每个元素最多被处理2次,所以时间复杂度为O(2*N) = O(N)。\\

    \item{\textbf{Solution}}\\
	\item{Hash} : \fbox{时间复杂度O($n$) , 空间复杂度O($n$) }\\
		\begin{lstlisting}
class Solution {
	public:
		string minWindow(string s, string t) {
			if(s.empty() || t.empty())
				return "";

			//出现的次数 + 存在性
			int times[256] = {0};
			bool isSeen[256] = {false};
			for(int i=0;i<t.size();++i) {
				++times[t[i]];
				isSeen[t[i]] = true;
			}

			int begin=0,end=-1;
			int count = t.size();
			int minLen = INT_MAX; 
			int minBegin = -1;
			//这里需要考虑当最后一个元素满足count==0,可是无法被处理的情况【if上面被处理后,不会进入下面】
			//首先,考虑在s后面插入辅助字符,这样必须选择t中没有出现过的字符,不具有扩展性
			//其次,考虑将end后移一位。
			int slen = s.size();
			while(begin < slen && end < slen) {
				if(count) {
					++end;
					//count只受t中元素影响
					if(isSeen[s[end]] && times[s[end]] > 0)
						--count;
					--times[s[end]];
				}else{
					if(end - begin + 1 < minLen) {
						minBegin = begin;
						minLen = end - begin + 1;
					}

					++times[s[begin]];
					if(isSeen[s[begin]] && times[s[begin]] > 0)
						++count;
					++begin;
				}
			}

			if(minLen == INT_MAX)
				return "";
			return s.substr(minBegin,minLen);
		}
};

		\end{lstlisting}

\end{description}

