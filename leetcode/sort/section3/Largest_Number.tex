    
\begin{description}
    \item{\textbf{问题}}: \\Given a list of non negative integers, arrange them such that they form the largest number. \textit{(leetcode 179)}
	\item{\textbf{举例}}:
given [3, 30, 34, 5, 9], the largest formed number is 9534330.
	\item{\textbf{Note}}:
The result may be very large, so you need to return a string instead of an integer.
    \item{\textbf{排序}} : \fbox{时间复杂度O(nlgn) , 空间复杂度O(n*m)}
    \\这题最主要是对元素进行排序,排序的关键是如何判断两个元素"谁大谁小",判断的标准是ab与ba大小对比.另外需要注意的是大数据下该用vector/string这样来操作.
    \begin{lstlisting}
static int march_bucket(int val, int pos){
	return (val / ((int)pow(10,pos))) % 10;
}

class mysort{
	public:
	bool cmp(vector<int> &a, vector<int> &b){
		int i = a.size() - 1;
		for(; i >= 0 && a[i] == b[i]; i--)
			;
		return i < 0? true : a[i] > b[i];
	}

	// return true if x > y
	bool operator()(int x, int y){
		if(x == 0 || y == 0)	return x > y;
		int i = 9, j = 9;
		vector<int> a(10,0), b(10,0);
		while(i >= 0){
			a[i] = Solution::march_bucket(x, i);
			b[i] = Solution::march_bucket(y, i);
			i--;
		}
		i = 9;
		while(i >= 0 && a[i] == 0)
			i--;
		while(j >= 0 && b[j] == 0)
			j--;
		vector<int> ab, ba;
		ab.insert(ab.end(), b.begin(), b.begin() + j +1);
		ab.insert(ab.end(), a.begin(), a.begin() + i + 1);
		ba.insert(ba.end(), a.begin(), a.begin() + i + 1);
		ba.insert(ba.end(), b.begin(), b.begin() + j +1);
		return cmp(ab , ba);
	}
};

string genString(int ele){
	string ret;
	int i = 9;
	bool start = false;
	while(i >= 0){
		int cur = march_bucket(ele, i);
		if(cur != 0 && !start)	start = true;
		if(start)	ret.push_back(cur + '0');
		i--;
	}
	if(!start)	return "0";
	return ret;
}

string largestNumber(vector<int> &num) {
	sort(num.begin(), num.end(), mysort());
	string result;
	for(auto ele : num){
		if(ele == 0 && result.empty())
			return "0";
		else
			result = result + genString(ele);
	}
	return result;
}
    \end{lstlisting}
\end{description}
