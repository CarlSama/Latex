    
\begin{description}
    \item{\textbf{问题}}:\\
Given a set of non-overlapping intervals, insert a new interval into the intervals (merge if necessary).\\
You may assume that the intervals were initially sorted according to their start times.\\
 \textit{(leetcode 57)}
	\item{\textbf{举例}}:\\
Given intervals [1,3],[6,9], insert and merge [2,5] in as [1,5],[6,9].\\
Given [1,2],[3,5],[6,7],[8,10],[12,16], insert and merge [4,9] in as [1,2],[3,10],[12,16].\\
This is because the new interval [4,9] overlaps with [3,5],[6,7],[8,10].\\
    \item{\textbf{排序}} : \fbox{时间复杂度O(nlgn) , 空间复杂度O(1)}
    \\先排序,然后找到插入点,插入进去.
    \begin{lstlisting}
vector<Interval> insert(vector<Interval> &intervals, Interval newInterval) {
	vector<Interval> result;
	Interval cur = newInterval;
	int pos = 0;
	bool isMerge = false;
	while(pos < intervals.size()){
		if(isMerge || cur.start > intervals[pos].end){
			result.push_back(intervals[pos]);
			pos++;
		}else{
			if(cur.end >= intervals[pos].start){
				cur.start = min(cur.start, intervals[pos].start);
				while(pos < intervals.size() && cur.end >= intervals[pos].start){
					cur.end = max(intervals[pos].end, cur.end);
					pos++;
				}
			}
			result.push_back(cur);
			isMerge = true;
		}
	}
	if(!isMerge)	result.push_back(cur);
	return result;
}
    \end{lstlisting}
\end{description}
