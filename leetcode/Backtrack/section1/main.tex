    
\qquad 回溯算法实际上一个类似枚举的搜索尝试过程,主要是在搜索尝试过程中寻找问题的解,当发现已不满足求解条件时,就“回溯”返回,尝试别的路径。

\qquad回溯法是一种选优搜索法,按选优条件向前搜索,以达到目标。但当探索到某一步时,发现原先选择并不优或达不到目标,就退回一步重新选择,这种走不通就退回再走的技术为回溯法,而满足回溯条件的某个状态的点称为“回溯点”。

\qquad在包含问题的所有解的解空间树中,按照深度优先搜索的策略,从根结点出发深度探索解空间树。当探索到某一结点时,要先判断该结点是否包含问题的解,如果包含,就从该结点出发继续探索下去,如果该结点不包含问题的解,则逐层向其祖先结点回溯。(其实回溯法就是对隐式图的深度优先搜索算法)。

\qquad有时候需要对搜索空间树进行剪枝,以加快回溯的速度.

