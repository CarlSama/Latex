    
\begin{description}
    \item{\textbf{问题}}:\\
Given a collection of candidate numbers (C) and a target number (T), find all unique combinations in C where the candidate numbers sums to T.\\
Each number in C may only be used once in the combination.\\
\textit{(leetcode 40)}
    \item{\textbf{Note}}:\\
All numbers (including target) will be positive integers.\\
Elements in a combination (a1, a2, … , ak) must be in non-descending order. (ie, a1 ≤ a2 ≤ … ≤ ak).\\
The solution set must not contain duplicate combinations.\\
    \item{\textbf{举例}}:\\
For example, given candidate set 10,1,2,7,6,1,5 and target 8, \\
A solution set is: \\
(1, 7) \\
(1, 2, 5)\\ 
(2, 6) \\
(1, 1, 6)\\ 
    \item{\textbf{回溯}} : \fbox{时间复杂度O($2^n$) , 空间复杂度O(n)}
    \\这题就是对于每个值包含还是不包含两种选择,然后进行回溯算法.这里需要注意的就是遇到一个值有多个的情况,这里通常最简单的处理办法就是进行浓缩,把原数组改成一个个点对,例如(1,1,1,2,3,3)改成((1,3),(2,1),(3,2))这样再每次选择时候消耗一下记数,这样就不会出现重复的结果了.
    \begin{lstlisting}
void backtrack(vector<int> &key, vector<int> &count, int target,
			int pos, vector<vector<int> > &result, vector<int> &track){
	if(target == 0){
		result.push_back(track);
		return;
	}
	if(pos == -1)	return;
	if(count[pos] > 0 && key[pos] <= target){
		track.push_back(key[pos]);
		count[pos]--;
		backtrack(key, count, target - key[pos], pos, result, track);
		count[pos]++;
		track.pop_back();
	}
	backtrack(key, count, target, pos-1, result, track);
}

vector<vector<int> > combinationSum2(vector<int> &num, int target) {
	sort(num.begin(), num.end(), greater<int>());
	vector<int> key, count, track;
	vector<vector<int> > result;
	if(num.size() == 0)	return result;
	int last = 0;
	for(int i = 1; i < num.size(); i++){
		if(num[i] != num[i-1]){
			key.push_back(num[last]);
			count.push_back(i - last);
			last = i;
		}
	}
	key.push_back(num[last]);
	count.push_back(num.size() - last);
	backtrack(key, count, target, key.size() - 1, result, track);
	return result;
}
    \end{lstlisting}
\end{description}
