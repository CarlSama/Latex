    
\begin{description}
    \item{\textbf{问题}}:\\
Given a collection of integers that might contain duplicates, nums, return all possible subsets.\\
\textit{(leetcode 90)}
    \item{\textbf{举例}}:\\
If nums = (1,2,2), a solution is:\\
\\
(\\
  (2),\\
  (1),\\
  (1,2,2),\\
  (2,2),\\
  (1,2),\\
  ()\\
)
    \item{\textbf{Note}}:\\
Elements in a subset must be in non-descending order.\\
The solution set must not contain duplicate subsets.\\
    \item{\textbf{回溯}} : \fbox{时间复杂度O($2^n$) , 空间复杂度O(n)}
    \\这题唯一的区别就是含有重复元素,对于这种问题还是那个老办法,先统计重复元素,制作成(元素,记数)对,然后再回溯搜索.
    \begin{lstlisting}
void backtrack(vector<vector<int> > &result, vector<int> &track, vector<int> & data,
				vector<int> &count, int pos, int n){
	if(pos == n){
		result.push_back(track);
		return;
	}
	for(int i = 0; i < count[pos]; i++){
		track.push_back(data[pos]);
		backtrack(result, track, data, count, pos+1, n);
	}
	for(int i = 0; i < count[pos]; i++){
		track.pop_back();
	}
	backtrack(result, track, data, count, pos+1, n);
}

vector<vector<int> > subsetsWithDup(vector<int> &S) {
	sort(S.begin(), S.end());
	vector<int> data, count;
	int last = 0;
	for(int i = 1; i < S.size(); i++){
		if(S[i] != S[i-1]){
			data.push_back(S[last]);
			count.push_back(i - last);
			last = i;
		}
	}
	data.push_back(S[last]);
	count.push_back(S.size() - last);
	vector<vector<int> > result;
	vector<int> track;
	backtrack(result, track, data, count, 0, data.size());
	return result;
}
    \end{lstlisting}
\end{description}
