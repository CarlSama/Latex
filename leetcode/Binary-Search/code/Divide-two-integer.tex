    
\begin{description}
    \item{\textbf{Question}}:\\%textbf加粗
		Divide two integers without using multiplication, division and mod operator.\\
		If it is overflow, return MAX\_INT.\\

    \item{\textbf{Anslysis}}\\
		假设被除数为s,除数为t,则s和t之间存在$s/t=k$.\\
		=>$s = k \ast  t = ((0/1)\ast 2^{m} + (0/1)\ast2^{m-1} + .... + (0/1)\ast2^{0}) \ast t$\\
		我们可以使用移位操作来判断每个位置的存在性.\\
		这里的性质说明了,任意正整数都可以表示为$2 ^ n$的和,而确定存在性的过程可以使用移位操作来二分的处理.\\
		这题目无法使用*,所以需要对t做移位操作来模拟乘二的行为.\\
		要处理0为除数,0为被除数,有正有负,除不尽,能除尽的情况\\
		在进行处理时,推荐使用long long类型,因为int类型在临界情况下会溢出,造成死循环.\\

    \item{\textbf{Solution}}\\
	\item{Binary Search} : \fbox{时间复杂度O($lgn$) , 空间复杂度O($1$)}\\
		\begin{lstlisting}
class Solution {
	public:
		int divide(int dividend, int divisor) {
			if(divisor == 0)	return INT_MAX;
			if(dividend == 0)	return 0;

			int flag = (dividend>0&&divisor<0 || dividend<0&&divisor>0)?-1:1;
			long long ldividend = abs((long long)dividend);
			//注意啊!无法使用int来接受
			long long ldivisor = abs((long long)divisor);
			if(ldividend < ldivisor)	return 0;
			if(ldividend == ldivisor)	return flag;

			long long result = 0;
			//在处理INT_MIN/1时会造成result(int)溢出
			while(ldividend >= ldivisor){
				long long tmpRes = 1;
				long long moreDivisor = ldivisor;
				//这里需要使用long long
				//不然可能造成溢出后进入死循环!
				while((moreDivisor << 1) <= ldividend){
					moreDivisor <<= 1;	tmpRes <<= 1;
				}
				result += tmpRes;
				ldividend -= moreDivisor;
			}
			if(flag == -1){
				if(result >= -1 * (long long)(INT_MIN))
					return INT_MIN;
				else
					return flag * result;
			}else{
				if(result >= (long long)INT_MAX)
					return INT_MAX;
				else
					return flag *result;
			}
		}
};
		\end{lstlisting}

	\item{Binary Search} : \fbox{时间复杂度O($1$)? , 空间复杂度O($1$)}\\
		因为结果为int类型,所以只需要考虑32位的移动情况,所以可以分次遍历divisor所能右移的情况.\\
		注意,需要从31到0,\\
		\begin{lstlisting}
		int divide(int dividend, int divisor) {
			if(divisor == 0)	return INT_MAX;
			if(dividend == 0)	return 0;

			int flag = (dividend>0&&divisor<0 || dividend<0&&divisor>0)?-1:1;
			long long ldividend = abs((long long)dividend);
			//注意啊!无法使用int来接受
			long long ldivisor = abs((long long)divisor);
			if(ldividend < ldivisor)	return 0;
			if(ldividend == ldivisor)	return flag;

			long long result = 0;
			for(int i=31;i>=0;--i){
				if((ldivisor<<i) <= ldividend){
					result += ((long long)1)<<i;
					//需要使用(long long)1,
					//否则1<<i得到int后转换为long long 
					ldividend -= (ldivisor<<i);
				}
			}
			if(flag == -1){
				if(result >= -1 * (long long)(INT_MIN))
					return INT_MIN;
				else
					return flag * result;
			}else{
				if(result >= (long long)INT_MAX)
					return INT_MAX;
				else
					return flag *result;
			}
		}
		\end{lstlisting}

\end{description}

