\documentclass{beamer}
\usetheme{Frankfurt}

\usepackage{CJK}
\hypersetup{CJKbookmarks=true} %解决section不能使用中文的问题

%%树
\usepackage{graphics,graphicx}
\usepackage{pstricks,pst-node,pst-tree}
\usepackage{caption}
\setbeamertemplate{caption}[numbered]
\def\dedge{\ncline[linestyle=dashed]}
%%

\begin{document} %申明文档的开始
\begin{CJK}{UTF8}{gbsn}     %CJK:支持中文

	\title{AVL树}
	\author{胡庆海}
	\date{}

    \begin{frame}
		\titlepage
    \end{frame}

    \begin{frame}
        \frametitle{目录}
		\tableofcontents
    \end{frame}

	\section{AVL树的定义}
	\begin{frame}
		\frametitle{AVL树的定义}
		AVL树是高度平衡的而二叉搜索树,且AVL树中任何节点的两个子树的高度最大差为1
	\end{frame}

	\section{AVL树的性质}
	\begin{frame}
		\frametitle{AVL树的性质}
		\begin{itemize}
			\item AVL树高度平衡,插入/删除/查找时间复杂度为$O(lg_n)$
		\end{itemize}
	\end{frame}

	\section{AVL树的操作}
	\begin{frame}
		\frametitle{AVL树的操作}
		\begin{itemize}
			\item 查找操作,同BST查找,不会打破树的平衡
			\item 插入操作,同BST查找,可能会打破树的平衡,需要调整
			\item 删除操作,同BST查找,可能会打破树的平衡,需要调整
		\end{itemize}
	\end{frame}

	\section{AVL树的调整}
	\begin{frame}
		\frametitle{AVL树的调整}
		AVL树由于插入或者删除会导致不平衡,主要有以下四种情况(设不平衡节点为观察节点)
		\begin{itemize}
		 \item[1] LL: 观察节点左孩子的左孩子的高度过高,使得节点左孩子的高度比右孩子的高度高2度 
		 \item[2] RR: 观察节点右孩子的右孩子的高度过高,使得节点右孩子的高度比左孩子的高度高2度 
		 \item[3] LR: 观察节点左孩子的右孩子的高度过高,使得节点左孩子的高度比右孩子的高度高2度 
		 \item[4] RL: 观察节点右孩子的左孩子的高度过高,使得节点右孩子的高度比左孩子的高度高2度 
		\end{itemize}
	\end{frame}

	\begin{frame}
		\frametitle{LL情况}
		LL情况下,左左子树高度过高,则只需要右旋观察节点,这样就可以恢复当前局部树的平衡了,剩下的只需要向上递归调整就可以了
			
		\begin{columns} %排版

		\pause
		\begin{column}{0.5\textwidth}
		\centering
		\begin{figure}
		\pstree[levelsep=25pt,treesep=6pt]{\Tcircle[fillcolor=lightgray,fillstyle=solid]{1}}
		{% levelsep指不同层之间距离, treesep兄弟之间距离
		 \pstree{\Tcircle[fillcolor=lightgray,fillstyle=solid]{2}}{
			\Tr{\psframebox[framearc=.5]{h+1}}
			\Tr{\psframebox[framearc=.5]{h}}
		 }
		 \Tr{\psframebox[framearc=.5]{h}}
		}
		\caption{调整之前}
		\end{figure}
		\end{column}

		\pause
		\begin{column}{0.5\textwidth}
		\centering
		\begin{figure}

		\pstree[levelsep=25pt,treesep=6pt]{\Tcircle[fillcolor=lightgray,fillstyle=solid]{2}}
		{% levelsep指不同层之间距离, treesep兄弟之间距离
		 \Tr{\psframebox[framearc=.5]{h+1}}
		 \pstree{\Tcircle[fillcolor=lightgray,fillstyle=solid]{1}}{
			\Tr{\psframebox[framearc=.5]{h}}
			\Tr{\psframebox[framearc=.5]{h}}
		 }
		}

		\caption{调整之后}
		\end{figure}
		\end{column}
	
		\end{columns}
	\end{frame}

	\begin{frame}
		\frametitle{RR情况}
		RR情况和LL情况类似,只需要左旋观察节点,这样就可以恢复当前局部树的平衡了,剩下的只需要向上递归调整就可以了
			
		\begin{columns} %排版

		\pause
		\begin{column}{0.5\textwidth}
		\centering
		\begin{figure}
		\pstree[levelsep=25pt,treesep=6pt]{\Tcircle[fillcolor=lightgray,fillstyle=solid]{1}}
		{% levelsep指不同层之间距离, treesep兄弟之间距离
		 \Tr{\psframebox[framearc=.5]{h}}
		 \pstree{\Tcircle[fillcolor=lightgray,fillstyle=solid]{2}}{
			\Tr{\psframebox[framearc=.5]{h}}
			\Tr{\psframebox[framearc=.5]{h+1}}
		 }
		}
		\caption{调整之前}
		\end{figure}
		\end{column}

		\pause
		\begin{column}{0.5\textwidth}
		\centering
		\begin{figure}

		\pstree[levelsep=25pt,treesep=6pt]{\Tcircle[fillcolor=lightgray,fillstyle=solid]{2}}
		{% levelsep指不同层之间距离, treesep兄弟之间距离
		 \pstree{\Tcircle[fillcolor=lightgray,fillstyle=solid]{1}}{
			\Tr{\psframebox[framearc=.5]{h}}
			\Tr{\psframebox[framearc=.5]{h}}
		 }
		 \Tr{\psframebox[framearc=.5]{h+1}}
		}

		\caption{调整之后}
		\end{figure}
		\end{column}
	
		\end{columns}
	\end{frame}

	\begin{frame}
		\frametitle{LR情况}
			以左孩子为旋转点左旋,再以观察节点为旋转点右旋
		\begin{columns} %排版

		\pause
		\begin{column}{0.3\textwidth}
		\centering
		\begin{figure}
		\pstree[levelsep=25pt,treesep=6pt]{\Tcircle[fillcolor=lightgray,fillstyle=solid]{1}}
		{% levelsep指不同层之间距离, treesep兄弟之间距离
		 \pstree{\Tcircle[fillcolor=lightgray,fillstyle=solid]{2}}{
			\Tr{\psframebox[framearc=.5]{h}}
		 	\pstree{\Tcircle[fillcolor=lightgray,fillstyle=solid]{3}}{
				\Tr{\psframebox[framearc=.5]{h}}
				\Tr{\psframebox[framearc=.5]{h-1}}
		 	}
		 }
		 \Tr{\psframebox[framearc=.5]{h}}
		}
		\caption{调整之前}
		\end{figure}
		\end{column}

		\pause
		\begin{column}{0.3\textwidth}
		\centering
		\begin{figure}

		\pstree[levelsep=25pt,treesep=6pt]{\Tcircle[fillcolor=lightgray,fillstyle=solid]{1}}
		{% levelsep指不同层之间距离, treesep兄弟之间距离
		 \pstree{\Tcircle[fillcolor=lightgray,fillstyle=solid]{3}}{
		 	\pstree{\Tcircle[fillcolor=lightgray,fillstyle=solid]{2}}{
				\Tr{\psframebox[framearc=.5]{h}}
				\Tr{\psframebox[framearc=.5]{h}}
		 	}
			\Tr{\psframebox[framearc=.5]{h-1}}
		 }
		 \Tr{\psframebox[framearc=.5]{h}}
		}

		\caption{左旋左孩子}
		\end{figure}
		\end{column}

		\pause
		\begin{column}{0.3\textwidth}
		\centering
		\begin{figure}

		\pstree[levelsep=25pt,treesep=6pt]{\Tcircle[fillcolor=lightgray,fillstyle=solid]{3}}
		{% levelsep指不同层之间距离, treesep兄弟之间距离
		 \pstree{\Tcircle[fillcolor=lightgray,fillstyle=solid]{2}}{
			\Tr{\psframebox[framearc=.5]{h}}
			\Tr{\psframebox[framearc=.5]{h}}
		 }
		 \pstree{\Tcircle[fillcolor=lightgray,fillstyle=solid]{1}}{
			\Tr{\psframebox[framearc=.5]{h-1}}
			\Tr{\psframebox[framearc=.5]{h}}
		}
		}

		\caption{右旋观察节点}
		\end{figure}
		\end{column}
	
		\end{columns}
	\end{frame}

	\begin{frame}
		\frametitle{RL情况}
			以右孩子为旋转点右旋,再以观察节点为旋转点左旋
		\begin{columns} %排版

		\pause
		\begin{column}{0.3\textwidth}
		\centering
		\begin{figure}
		\pstree[levelsep=25pt,treesep=6pt]{\Tcircle[fillcolor=lightgray,fillstyle=solid]{1}}
		{% levelsep指不同层之间距离, treesep兄弟之间距离
		 \Tr{\psframebox[framearc=.5]{h}}
		 \pstree{\Tcircle[fillcolor=lightgray,fillstyle=solid]{2}}{
		 	\pstree{\Tcircle[fillcolor=lightgray,fillstyle=solid]{3}}{
				\Tr{\psframebox[framearc=.5]{h}}
				\Tr{\psframebox[framearc=.5]{h-1}}
		 	}
			\Tr{\psframebox[framearc=.5]{h}}
		 }
		}
		\caption{调整之前}
		\end{figure}
		\end{column}

		\pause
		\begin{column}{0.3\textwidth}
		\centering
		\begin{figure}

		\pstree[levelsep=25pt,treesep=6pt]{\Tcircle[fillcolor=lightgray,fillstyle=solid]{1}}
		{% levelsep指不同层之间距离, treesep兄弟之间距离
		 \Tr{\psframebox[framearc=.5]{h}}
		 \pstree{\Tcircle[fillcolor=lightgray,fillstyle=solid]{3}}{
			\Tr{\psframebox[framearc=.5]{h}}
		 	\pstree{\Tcircle[fillcolor=lightgray,fillstyle=solid]{2}}{
				\Tr{\psframebox[framearc=.5]{h-1}}
				\Tr{\psframebox[framearc=.5]{h}}
		 	}
		 }
		}

		\caption{右旋右孩子}
		\end{figure}
		\end{column}

		\pause
		\begin{column}{0.3\textwidth}
		\centering
		\begin{figure}

		\pstree[levelsep=25pt,treesep=6pt]{\Tcircle[fillcolor=lightgray,fillstyle=solid]{3}}
		{% levelsep指不同层之间距离, treesep兄弟之间距离
		 \pstree{\Tcircle[fillcolor=lightgray,fillstyle=solid]{1}}{
		 	\Tr{\psframebox[framearc=.5]{h}}
			\Tr{\psframebox[framearc=.5]{h}}
		 }
		 \pstree{\Tcircle[fillcolor=lightgray,fillstyle=solid]{2}}{
				\Tr{\psframebox[framearc=.5]{h-1}}
				\Tr{\psframebox[framearc=.5]{h}}
		 }
		}

		\caption{左旋观察节点}
		\end{figure}
		\end{column}
	
		\end{columns}
	\end{frame}

\end{CJK}
\end{document}
