    
\begin{description}
    \item{\textbf{Question}}:\\%textbf加粗
		Given a list, rotate the list to the right by k places, where k is non-negative.

    \item{\textbf{Anslysis}}\\
	使用merge sort. 递归的方式就是按照merge sort的标准样式进行,merge的过程也使用递归的方法.在迭代的方法中需要申请额外的空间来存储.

    \item{\textbf{Solution}}
	\item{递归} : \fbox{时间复杂度O(n) , 空间复杂度O(1) }\\
		\begin{lstlisting}
public class Solution {
    //merge two sorted list, return result head
    public ListNode merge(ListNode h1, ListNode h2){
        if(h1 == null){
            return h2;
        }
        if(h2 == null){
            return h1;
        }

        if(h1.val < h2.val){
            h1.next = merge(h1.next, h2);
            return h1;
        }
        else{
            h2.next = merge(h1, h2.next);
            return h2;
        }

    }

    public ListNode sortList(ListNode head) {
        //bottom case
        if(head == null){
            return head;
        }
        if(head.next == null){
            return head;
        }

        //p1 move 1 step every time, p2 move 2 step every time, pre record node before p1
        ListNode p1 = head;
        ListNode p2 = head;
        ListNode pre = head;

        while(p2 != null && p2.next != null){
            pre = p1;
            p1 = p1.next;
            p2 = p2.next.next;
        }
        //change pre next to null, make two sub list(head to pre, p1 to p2)
        pre.next = null;

        //handle those two sub list
        ListNode h1 = sortList(head);
        ListNode h2 = sortList(p1);

        return merge(h1, h2);

    }

}
		\end{lstlisting}
	\item{迭代} : \fbox{时间复杂度O(n) , 空间复杂度O(1) }\\
		\begin{lstlisting}
 ListNode *sortList(ListNode *pHead){
    if(!pHead || !pHead->next)  return pHead;
    vector<ListNode*> counter;
    for(int i = 0; i < 64; i++)
        counter.push_back(NULL);
    ListNode *carry;
    ListNode *pos = pHead;
    int fill = 0;
    while(pos){
        carry = new ListNode(pos->val);
        pos = pos->next;
        int i = 0;
        for(i = 0; i < fill && counter[i]; i++){
            carry = MergeSortedList(carry, counter[i]);
            counter[i] = NULL;
        }
        counter[i] = carry;
        if(i == fill) fill++;
    }
    for(int i = 1; i < fill; i++){
        counter[i] = MergeSortedList(counter[i-1], counter[i]);
    }
    return counter[fill-1];
}
		\end{lstlisting}
\end{description}

