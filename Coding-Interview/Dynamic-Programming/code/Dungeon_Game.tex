    
\begin{description}
    \item{\textbf{Question}}:\\%textbf加粗
		The demons had captured the princess (P) and imprisoned her in the bottom-right corner of a dungeon. The dungeon consists of M x N rooms laid out in a 2D grid. Our valiant knight (K) was initially positioned in the top-left room and must fight his way through the dungeon to rescue the princess.\\
		The knight has an initial health point represented by a positive integer. If at any point his health point drops to 0 or below, he dies immediately.\\
		Some of the rooms are guarded by demons, so the knight loses health (negative integers) upon entering these rooms; other rooms are either empty (0's) or contain magic orbs that increase the knight's health (positive integers).\\
		In order to reach the princess as quickly as possible, the knight decides to move only rightward or downward in each step.\\

    \item{\textbf{Anslysis}}\\
		这是一道非常好的题目.有的问题有多重子问题的划分方法,在其中有的划分方法中会存在有后效性;这是我们应该尝试寻找无后效性且具有最右子结构的划分.\\
		对于这个题目,如果我们将dp[m,n]定义为从[0,0]走到[m,n]时所需的最少血量,这时就有一个问题,每次的决策即依赖于之前最小血量的路径,又以来与从不通路径走后当前剩余的血量,这个当前剩余的血量会影响到对之后路径的选择,所以这个角度分析问题是有后效性的:(\\
		我们将dp[i,j]看做是从[i,j]开始到重点时所需的最少血量,也就是说[i,j]总是被认为是起点,这样就不存在剩余血量的说法了,也就没有了后效性.:)

    \item{\textbf{Solution}}\\
	\item{1. DFS} : \fbox{时间复杂度O($2^n$) , 空间复杂度O(n)  TLE}\\
		\begin{lstlisting}
class Solution {
	private:
		int cols,rows;
	public:
		int calculateMinimumHP(vector<vector<int> >& dungeon) {
			if(dungeon.size()==0 || dungeon[0].size()==0)	return 0;
			cols = dungeon.size();	rows = dungeon[0].size();
			vector<vector<int> > dp(cols,vector<int>(rows,0));
			dp[cols-1][rows-1] = max(1 - dungeon[cols-1][rows-1],1);
			for(int i = cols-2;i>=0;--i)
				dp[i][rows-1] = max(dp[i+1][rows-1] - dungeon[i][rows-1], 1);
			for(int i = rows-2;i>=0;--i)
				dp[cols-1][i] = max(dp[cols-1][i+1] - dungeon[cols-1][i], 1);
			for(int i = cols-2;i>=0;--i){
				for(int j=rows-2;j>=0;--j){
					dp[i][j]=min(max(dp[i+1][j]-dungeon[i][j],1),max(dp[i][j+1]-dungeon[i][j],1));
				}
			}
			return dp[0][0];
		}
};
		\end{lstlisting}

	\item{2. DP} : \fbox{时间复杂度O($n^2$) , 空间复杂度O(n)}\\
		最优子结构的性质:如果s[1,k]可以被dict划分,s[k,j]又存在于dict中,那么自然s[i,j]就可以被dict划分;\\
		子问题重叠:对于s="bbcandaa"和dict=["b","bc","and","anda","a","aa"]的情况,在计算and时,and之前的可划分性(p[0,2])被计算;在之后处理anda时,anda之前的可划分性(p[0,2])又会被需要.\\
		无后效性:对于某个位置的决策只依赖于前一位置是否可划分,并且只依赖于之前的某一个决策(无并行后效性);也无后方后效性.\\

    \begin{lstlisting}
class Solution {
	public:
		bool wordBreak(string s, unordered_set<string>& wordDict) {
			vector<bool> dp(s.size()+1,false);//dp[i] -> [0,i-1]
			dp[0] = true;
			for(int i=0;i<s.size();++i){
				for(int j=i;j>=0;--j){//[j,i]
					if(dp[j+1-1]){//[0,j-1] == true;
						string subStr = s.substr(j,i-j+1);
						if(wordDict.find(subStr)!=wordDict.end()){
							dp[i+1] = true;
							break;
						}
					}
				}
			}
			return dp[s.size()];
	    }
};
    \end{lstlisting}
	\textit{只需一个维度即可解决问题}
\end{description}

