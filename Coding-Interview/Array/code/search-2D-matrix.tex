    
\begin{description}
    \item{\textbf{Question}}:\\%textbf加粗
		Write an efficient algorithm that searches for a value in an m x n matrix. This matrix has the following properties:\\
		Integers in each row are sorted from left to right.\\
		The first integer of each row is greater than the last integer of the previous row.\\

    \item{\textbf{Example}}\\
		{1,   3,  5,  7}\\
		{10, 11, 16, 20}\\
		{23, 30, 34, 50}\\
		and target = 3\\

		return true\\
    \item{\textbf{Anslysis}}\\
		因为是已排序的对象,所以参考二分的方法\\
		1)首先比较最右侧(最大值们),来确定target所在的行号;然后在那一行进行二分查找\\
		2)可以将二位矩阵看做一维数组。但是会有较高cache miss率,而且总长度(m*n)可能会对int溢出\\

    \item{\textbf{Solution}}\\
	\item{} : \fbox{时间复杂度O(lg(m*n)) , 空间复杂度O(1) }\\
		\begin{lstlisting}
class Solution {
	public:
		bool searchMatrix(vector<vector<int> >& matrix, int target) {
			int col = matrix.size();
			if(col==0)return false;
			int row = matrix[0].size();

			int top=0,bottom=col-1;
			while(top < bottom){
				int mid = top + (bottom-top)/2;
				if(matrix[mid][row-1]==target)	
					return true;
				else if(matrix[mid][row-1] > target)	
					bottom = mid;
				else	
					top=mid+1;
			}
			int level = top;
			top=0,bottom=row-1;
			while(top <= bottom){
				int mid = top + (bottom - top) / 2;
				if(matrix[level][mid] == target)
					return true;
				else if(matrix[level][mid] > target)
					bottom = mid - 1;
				else 
					top = mid + 1;
			}
			return false;
	}
};
		\end{lstlisting}
	\item{} : \fbox{时间复杂度O(lgm + lgn) = O(lg(m*n)) , 空间复杂度O(1) }\\
		\begin{lstlisting}

class Solution {
public:
    bool searchMatrix(vector<vector<int> > &matrix, int target) {
        int n = matrix.size();
        int m = matrix[0].size();
        int l = 0, r = m * n - 1;
        while (l != r){
            int mid = (l + r - 1) >> 1;
            if (matrix[mid / m][mid % m] < target)
                l = mid + 1;
            else 
                r = mid;
        }
        return matrix[r / m][r % m] == target;
    }
};
		\end{lstlisting}
\end{description}

