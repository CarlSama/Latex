    
\begin{description}
    \item{\textbf{Question}}:\\%textbf加粗
		Given a collection of candidate numbers (C) and a target number (T), find all unique combinations in C where the candidate numbers sums to T.\\
		Each number in C may only be used once in the combination.\\
		Note:\\
		All numbers (including target) will be positive integers.\\
		Elements in a combination (a1, a2, … , ak) must be in non-descending order. (ie, a1 ≤ a2 ≤ … ≤ ak).\\
		The solution set must not contain duplicate combinations.\\

    \item{\textbf{Example}}\\
		For example, given candidate set 10,1,2,7,6,1,5 and target 8.\\
		A solution set is:\\
		1, 7\\
		1, 2, 5\\
		2, 6\\
		1, 1, 6\\

    \item{\textbf{Anslysis}}\\
		与combination sum类似,但是某个元素被选过后就不能被再次使用\\

    \item{\textbf{Solution}}\\
	\item{} : \fbox{时间复杂度O($n^2$) , 空间复杂度O(n) }\\
		\begin{lstlisting}
class Solution {
public:
    std::vector<std::vector<int> > combinationSum2(std::vector<int> &candidates, int target) {
        std::sort(candidates.begin(), candidates.end());
        std::vector<std::vector<int> > res;
        std::vector<int> combination;
        combinationSum2(candidates, target, res, combination, 0);
        return res;
    }
private:
    void combinationSum2(std::vector<int> &candidates, int target, std::vector<std::vector<int> > &res, std::vector<int> &combination, int begin) {
        if  (!target) {
            res.push_back(combination);
            return;
        }
        for (int i = begin; i != candidates.size() && target >= candidates[i]; ++i)
            if (i == begin || candidates[i] != candidates[i - 1]) {
                combination.push_back(candidates[i]);
                combinationSum2(candidates, target - candidates[i], res, combination, i + 1);
                combination.pop_back();
            }
    }
};		\end{lstlisting}

\end{description}

