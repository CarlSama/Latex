    
\begin{description}
    \item{\textbf{Question}}:\\%textbf加粗
		Given two sorted integer arrays nums1 and nums2, merge nums2 into nums1 as one sorted array.\\

    \item{\textbf{Anslysis}}\\
		反向拷贝num1中数据到尾部,空出空间来做排序操作\\
		OH, FUCK.我们可以从头部开始的插入,我们可以从尾部开始插入!这样就避免了为空出空间而做的移动操作\\

    \item{\textbf{Solution}}\\
	\item{1.From Front} : \fbox{时间复杂度O($n$) , 空间复杂度O(1) }\\
		\begin{lstlisting}
class Solution {
		int len1;
public:
		void  copy_to_last(vector<int> & nums1,int m){
				//we need backward-copy
				len1 = nums1.size();
				int from = m - 1,to = len1 - 1;
				while(from >= 0) {
						nums1[to] = nums1[from];
						--from;	--to;
				}
		}

	    void merge(vector<int>& nums1, int m, vector<int>& nums2, int n) {
				if(n == 0)	return; 

				if(nums1[m-1] <= nums2[0]) {//just paste to end of nums1
						for(int i=0 ; i<n ; ++i) {
								nums1[m + i] = nums2[i];
						}
						return;
				}

				copy_to_last(nums1, m);
				int mi = len1-m , ni = 0, idx = 0;
				while(mi < len1 && ni < n) {
						if(nums1[mi] < nums2[ni]) {
								nums1[idx] = nums1[mi];
								++mi;
						}else{
								nums1[idx] = nums2[ni];
								++ni;
						}
						++idx;
				}

				while(mi < len1){
					nums1[idx] = nums1[mi];
					++idx;	++mi;
				}
				while(ni < n){
					nums1[idx] = nums2[ni];
					++idx;	++ni;
				}
		}
};
		\end{lstlisting}

	\item{2.From Back} : \fbox{时间复杂度O($n$) , 空间复杂度O(1) }\\
		\begin{lstlisting}
class Solution {
public:
    void merge(int A[], int m, int B[], int n) {
        int i=m-1;
        int j=n-1;
        int k = m+n-1;
        while(i >=0 && j>=0)
        {
            if(A[i] > B[j])
                A[k--] = A[i--];
            else
                A[k--] = B[j--];
        }
        while(j>=0)
            A[k--] = B[j--];
    }
};
	\end{lstlisting}

\end{description}
