    
\begin{description}
    \item{\textbf{Question}}:\\%textbf加粗
		Given an array of non-negative integers, you are initially positioned at the first index of the array.\\
		Each element in the array represents your maximum jump length at that position.\\
		Your goal is to reach the last index in the minimum number of jumps.\\

    \item{\textbf{Example}}\\
		Given array A = [2,3,1,1,4]\\
		The minimum number of jumps to reach the last index is 2. (Jump 1 step from index 0 to 1, then 3 steps to the last index.)\\

    \item{\textbf{Anslysis}}\\
		我们从左向右的扫描数组,在某个位置i可以跳到位置i + steps[i],如果这个位置之前没有被到达,则到达该位置的最少跳数就为i + steps[i].\\
		使用替代法来证明:首先,初始时第一个位置所能达到的位置为0 + steps[0],所以从[0, 0+steps[0]]的最小跳数为1;\\
		在跳动的过程中,假设已求得达到某个位置i的最小跳数为minSteps[i],则从i向后跳所能到达的位置为[i+1, i+steps[i]],如果这个范围内存在j,j之前还没有到达,则到达j的最小跳数为minSteps[i] + 1。假设到达j的跳数小于minSteps[i]+1,则j之前一定已经被到达过了,与假设矛盾。\\

    \item{\textbf{Solution}}\\
	\item{} : \fbox{时间复杂度O(n) , 空间复杂度O(n) }\\
		\begin{lstlisting}
int jump(int A[], int n) {
    if(n == 0){
        return 0;
    }
    int maxReachPos = A[0];
    int curMaxReachPos = A[0];
    int curStep = 1;
    for(int i = 1; i <= min(n, maxReachPos); i++){
        curMaxReachPos = max(curMaxReachPos, i + A[i]);
        if(i == n - 1){
            return curStep;
        }
        if(i == maxReachPos){
            maxReachPos = curMaxReachPos;
            curStep++;
        }
    }
    return 0;
}		\end{lstlisting}

\end{description}

