    
\begin{description}
    \item{\textbf{Question}}:\\%textbf加粗
		Implement next permutation, which rearranges numbers into the lexicographically next greater permutation of numbers.\\
		If such arrangement is not possible, it must rearrange it as the lowest possible order (ie, sorted in ascending order).\\
		The replacement must be in-place, do not allocate extra memory.\\

    \item{\textbf{Example}}\\
		1,2,3 → 1,3,2\\
		3,2,1 → 1,2,3\\
		1,1,5 → 1,5,1\\

    \item{\textbf{Anslysis}}\\
		要依据当前的数值寻找到下一个大小的数值。\\
		从后向前寻找升序的截止点i,以及截止点前的位置i-1;在后面的升序中查找比val[i-1]大的第一个位置j,交换i-1与j,然后反转[i, end]\\

		可使用的stl为:reverse, upperboud,swap。

    \item{\textbf{Solution}}\\
	\item{} : \fbox{时间复杂度O(n) , 空间复杂度O(1) }\\
		\begin{lstlisting}
class Solution {
	public:
		void nextPermutation(vector<int> &num) 
		{
			if (num.empty()) return;

			// in reverse order, find the first number which is in increasing trend (we call it violated number here)
			int i;
			for (i = num.size()-2; i >= 0; --i)
			{
				if (num[i] < num[i+1]) break;
			}

			// reverse all the numbers after violated number
			reverse(begin(num)+i+1, end(num));
			// if violated number not found, because we have reversed the whole array, then we are done!
			if (i == -1) return;
			// else binary search find the first number larger than the violated number
			auto itr = upper_bound(begin(num)+i+1, end(num), num[i]);
			// swap them, done!
			swap(num[i], *itr);
		}
};
		\end{lstlisting}
\end{description}

