    
\begin{description}
    \item{\textbf{Question}}:\\%textbf加粗
		A peak element is an element that is greater than its neighbors.\\
		Given an input array where num[i] ≠ num[i+1], find a peak element and return its index.\\
		The array may contain multiple peaks, in that case return the index to any one of the peaks is fine.\\
		You may imagine that num[-1] = num[n] = -∞.\\

    \item{\textbf{Example}}\\
		In array [1, 2, 3, 1], 3 is a peak element and your function should return the index number 2.\\

    \item{\textbf{Anslysis}}\\
		如果使用O(n)的算法,则可以从左向右遍历求解\\
		我们来考虑在找到某个点时出现的情况:1.left < curr \&\& curr > right,则right为所求;\\
		2.left < curr \&\& curr < right,则必定存在一个peak位于curr的右侧;\\
		3.left > curr \&\& curr > right, 则必定存在一个peak位于curr的左侧;\\
		4.left > curr \&\& right \&\& curr,则curr的左侧和右侧必定分别存在一个peak;\\
		我们可以参照上面的分析,使用二分的方法来合理的移动left,right指针,从而定位到心仪的数据。这样就只需要O(lgn)的时间复杂度。\\

    \item{\textbf{Solution}}\\
	\item{} : \fbox{时间复杂度O(lgn) , 空间复杂度O(1) }\\
		\begin{lstlisting}
class Solution {
public:
    int findPeakElement(const vector<int> &num) 
    {
        int low = 0;
        int high = num.size()-1;

        while(low < high)
        {
            int mid1 = (low+high)/2;
            int mid2 = mid1+1;
            if(num[mid1] < num[mid2])
                low = mid2;
            else
                high = mid1;
        }
        return low;
    }
};		\end{lstlisting}

\end{description}

