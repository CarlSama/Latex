    
\begin{description}
    \item{\textbf{Question}}:\\%textbf加粗
		Given an array S of n integers, find three integers in S such that the sum is closest to a given number, target. Return the sum of the three integers. You may assume that each input would have exactly one solution.\\

    \item{\textbf{Example}}\\
		Given array S = {-1 2 1 -4}, and target = 1.\\
		The sum that is closest to the target is 2. (-1 + 2 + 1 = 2).\\

    \item{\textbf{Anslysis}}\\
		类似于3-sum\\

    \item{\textbf{Solution}}\\
	\item{} : \fbox{时间复杂度O($n^2$) , 空间复杂度O(1) }\\
		\begin{lstlisting}
int threeSumClosest(vector<int> &num, int target) {        
    vector<int> v(num.begin(), num.end()); // I didn't wanted to disturb original array.
    int n = 0;
    int ans = 0;
    int sum;

    sort(v.begin(), v.end());

    // If less then 3 elements then return their sum
    while (v.size() <= 3) {
        return accumulate(v.begin(), v.end(), 0);
    }

    n = v.size();

    /* v[0] v[1] v[2] ... v[i] .... v[j] ... v[k] ... v[n-2] v[n-1]
     *                    v[i]  <=  v[j]  <= v[k] always, because we sorted our array. 
     * Now, for each number, v[i] : we look for pairs v[j] & v[k] such that 
     * absolute value of (target - (v[i] + v[j] + v[k]) is minimised.
     * if the sum of the triplet is greater then the target it implies
     * we need to reduce our sum, so we do K = K - 1, that is we reduce
     * our sum by taking a smaller number.
     * Simillarly if sum of the triplet is less then the target then we
     * increase out sum by taking a larger number, i.e. J = J + 1.
     */
    ans = v[0] + v[1] + v[2];
    for (int i = 0; i < n-2; i++) {
        int j = i + 1;
        int k = n - 1;
        while (j < k) {
            sum = v[i] + v[j] + v[k];
            if (abs(target - ans) > abs(target - sum)) {
                ans = sum;
                if (ans == target) return ans;
            }
            (sum > target) ? k-- : j++;
        }
    }
    return ans;
}
		\end{lstlisting}

\end{description}

