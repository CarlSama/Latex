    
\begin{description}
    \item{\textbf{Question}}:\\%textbf加粗
		Write an efficient algorithm that searches for a value in an m x n matrix. This matrix has the following properties:\\
		Integers in each row are sorted from left to right.\\
		The first integer of each row is greater than the last integer of the previous row.\\

    \item{\textbf{Anslysis}}\\
		因为矩阵满足按序排列的性质,所以可使用二分查找.\\

    \item{\textbf{Solution}}\\
	\item{Binary Search} : \fbox{时间复杂度O($lg(m*n)$) , 空间复杂度O($1$)}\\
		\begin{lstlisting}
class Solution {
	public:
		bool searchMatrix(vector<vector<int> >& matrix, int target) {
			int col = matrix.size();
			if(col==0)return false;
			int row = matrix[0].size();

			int top=0,bottom=col-1;
			while(top < bottom){
				int mid = top + (bottom-top)/2;
				if(matrix[mid][row-1]==target)	
					return true;
				else if(matrix[mid][row-1] > target)	
					bottom = mid;
				else	
					top=mid+1;
			}
			int level = top;
			top=0,bottom=row-1;
			while(top <= bottom){
				int mid = top + (bottom - top) / 2;
				if(matrix[level][mid] == target)
					return true;
				else if(matrix[level][mid] > target)
					bottom = mid - 1;
				else 
					top = mid + 1;
			}
			return false;
	}
};
		\end{lstlisting}
	\item{Use STL} : \fbox{时间复杂度O($lg(m*n)$) , 空间复杂度O($1$)}\\
		\begin{lstlisting}
		vector<int> searchRange(vector<int>& nums, int target) {
			pair<vector<int>::iterator, vector<int>::iterator > bounds;
			bounds = std::equal_range(nums.begin(), nums.end(), target);
			if(bounds.first == nums.end() || *(bounds.first) != target)
				return vector<int>(2,-1);
			vector<int> res;
			res.push_back(bounds.first - nums.begin());	res.push_back(bounds.second - nums.begin() -1);
			return res;
		}
		vector<int> searchRange(vector<int>& nums, int target) {
			vector<int>::iterator lo = std::lower_bound(nums.begin(),nums.end(),target);//first elem not less than target
			vector<int>::iterator up = std::upper_bound(nums.begin(),nums.end(),target);//first elem great than target
			if(lo == nums.end() || *lo != target)
				return vector<int>(2,-1);
			vector<int> res;
			res.push_back(lo - nums.begin());	res.push_back(up - nums.begin() -1);
			return res;
		}
		\end{lstlisting}
\end{description}

