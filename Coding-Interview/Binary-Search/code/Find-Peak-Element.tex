    
\begin{description}
    \item{\textbf{Question}}:\\%textbf加粗
		A peak element is an element that is greater than its neighbors.\\
		Given an input array where $num[i] \neq num[i+1]$, find a peak element and return its index.\\
		The array may contain multiple peaks, in that case return the index to any one of the peaks is fine.\\
		You may imagine that $num[-1] = num[n] = -∞$.
    \item{\textbf{Example}}\\
		In array [1, 2, 3, 1], 3 is a peak element and your function should return the index number 2.
    \item{\textbf{Anslysis}}\\
		可以使用顺序查找的方法,也可以使用二分的方法.\\
		当使用二分的方法时,需要判断$mid$与$mid + 1$的关系.\\

    \item{\textbf{Solution}}\\
	\item{Binary Search} : \fbox{时间复杂度O($lgn$) , 空间复杂度O($1$)}\\
		\begin{lstlisting}
class Solution {
public:
    int findPeakElement(const vector<int> &num) 
    {
        int low = 0;
        int high = num.size()-1;

        while(low < high)
        {
            int mid1 = (low+high)/2;
            int mid2 = mid1+1;
            if(num[mid1] < num[mid2])
                low = mid2;
            else
                high = mid1;
        }
        return low;
    }
};
		\end{lstlisting}

	\item{Sequential Search} : \fbox{时间复杂度O($n$)? , 空间复杂度O($1$)}\\
		在使用顺序查找的方法时,不需要比较$a[i] a[i-1] a[i+1]$的关系,只需检测当前的值是否大于前一位置的值,如果大于,则继续寻找,表明现在处于升序阶段;如果小于,则可以确定这里是降序阶段的起始位置.\\
		\begin{lstlisting}
class Solution {
public:
    int findPeakElement(const vector<int> &num) {
        for(int i = 1; i < num.size(); i ++){
            if(num[i] < num[i-1]){
                return i-1;
            }
        }
        return num.size()-1;
    }
};
		\end{lstlisting}

\end{description}

