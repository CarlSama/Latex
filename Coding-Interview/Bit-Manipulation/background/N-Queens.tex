    
\begin{description}
    \item{\textbf{问题}}:\\
The n-queens puzzle is the problem of placing n queens on an n×n chessboard such that no two queens attack each other.\\
Given an integer n, return all distinct solutions to the n-queens puzzle.\\
Each solution contains a distinct board configuration of the n-queens' placement, where 'Q' and '.' both indicate a queen and an empty space respectively.\textit{(leetcode 51)}
    \item{\textbf{回溯}} : \fbox{时间复杂度O(n!), 空间复杂度O($n^2$)}
    \\其实时间复杂度应该远小于O(n!),这里只是确定一个上界.
    \begin{lstlisting}
bool isSafe(vector<pair<int, int> > &queens, int i, int j){
	for(int k = 0; k < queens.size(); k++){
		int x = queens[k].first, y = queens[k].second;
		if(x == i || y == j || abs(i - x) == abs(j - y))
			return false;
	}
	return true;
}

void backtrack(vector<vector<string> > &result, vector<pair<int, int> > &queens,
			vector<string> &track, int row, int row_num){
	if(row == row_num){
		result.push_back(track);
		return;
	}
	string lines(row_num, '.');
	for(int i = 0; i < row_num; i++){
		if(isSafe(queens, row, i)){
			queens.push_back(make_pair(row, i));
			lines[i] = 'Q';
			track.push_back(lines);
			backtrack(result, queens, track, row+1, row_num);
			track.pop_back();
			lines[i] = '.';
			queens.pop_back();
		}
	}
}

vector<vector<string> > solveNQueens(int n) {
	vector<vector<string> > result;
	if(n <= 0)	return result;
	vector<pair<int, int> > queens;
	vector<string> track;
	backtrack(result, queens, track, 0, n);
	return result;
}
    \end{lstlisting}
\end{description}
