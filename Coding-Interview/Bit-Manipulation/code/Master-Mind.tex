\begin{description}
    \item{\textbf{Question}}:\\
		The Game of Master Mind is played as follows:\\
		The computer has four slots containing balls that are red (R), yellow (Y), green (G) or blue (B). For example, the computer might have RGGB (e.g., Slot #1 is red, Slots #2 and #3 are green, Slot #4 is blue). \\
		You, the user, are trying to guess the solution. You might, for example, guess YRGB. When you guess the correct color for the correct slot, you get a “hit”. If you guess a color that exists but is in the wrong slot, you get a “pseudo-hit”. For example, the guess YRGB has 2 hits and one pseudo hit.\\
		For each guess, you are told the number of hits and pseudo-hits. Write a method that, given a guess and a solution, returns the number of hits and pseudo hits.

    \item{\textbf{Analysis}} \\
		首先的想法是通过遍历两组数据,可以得到hit的次数,以及非hit数据出现的个数,然后比较非hit数据来得到pseudo-hit的次数。\\
		能否不保存每个颜色出现的次数(当颜色种类增加时,所需的空间也就线性增加).因为颜色使用字母来表征,所以32位的int足够来记录颜色的存在性信息(hash的变种,使用位向量来表征存在性)。\\
		疑问点:某个单词只要出现就可以被认为是pseudo-hit吗?还是有次数限制?

    \item{\textbf{}} : \fbox{时间复杂度O(1), 空间复杂度O(1)}
    \begin{lstlisting}
public static class Result {
	public int hits;
	public int pseudoHits;
};
	
public static Result estimate(String guess, String solution) {
	Result res = new Result();
	int solution_mask = 0;
	for (int i = 0; i < 4; ++i) 
		solution_mask |= 1 << (1 + solution.charAt(i) - ‘A’);

	for (int i = 0; i < 4; ++i) {
		if (guess.charAt(i) == solution.charAt(i)) {
			++res.hits;
		} else if ((solution_mask & (1 << (1 + guess.charAt(i) - ‘A’))) >= 1) {
			++res.pseudoHits;
		}
	}
	return res;
}
    \end{lstlisting}
\end{description}
