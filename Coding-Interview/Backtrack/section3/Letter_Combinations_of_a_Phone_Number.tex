    
\begin{description}
    \item{\textbf{问题}}:\\
Given a digit string, return all possible letter combinations that the number could represent.\\
A mapping of digit to letters (just like on the telephone buttons) is given below.\\

\includegraphics{backtrack/section3/LetterCombinations.eps}

\textit{(leetcode 17)}
	\item{\textbf{举例}}:\\
Input:Digit string "23"\\
Output: ["ad", "ae", "af", "bd", "be", "bf", "cd", "ce", "cf"].\\
	\item{\textbf{Note}}:\\
Although the above answer is in lexicographical order, your answer could be in any order you want.
    \item{\textbf{回溯}} : \fbox{时间复杂度O($3^n$), 空间复杂度O(n)}
    \begin{lstlisting}
string table[] = {" ","", "abc", "def", "ghi", "jkl", "mno", "pqrs", "tuv", "wxyz"};

void dfs(vector<string> &result, string& digits, string& trace, int pos){
	if(pos == digits.size()){
		result.push_back(trace);
		return;
	}
	string& all = table[digits[pos] - '0'];
	for(int i = 0; i <  all.size(); i++){
		trace.push_back(all[i]);
		dfs(result, digits, trace, pos+1);
		trace.erase(trace.size() - 1);
	}
}

vector<string> letterCombinations(string digits) {
	vector<string> result;
	int n = digits.size();
	if(n == 0)	return result;
	string trace;
	dfs(result, digits, trace, 0);	
	return result;
}
    \end{lstlisting}
    \textit{}
\end{description}
