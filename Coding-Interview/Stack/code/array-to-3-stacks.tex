    
\begin{description}
    \item{\textbf{Question}}:\\%textbf加粗
		Describe how you could use a single array to implement three stacks.

    \item{\textbf{Anslysis}}\\
		1. 均分数组空间给三个堆栈使用(可使用的空间无法再增长)。\\
		2. 平等的使用数组空间,每个新加入元素都包含指向前一元素的指针(当删除某个元素后,这个空间可能无法及时被其他堆栈使用)。可以维护一个free-list来存储所有可用的空间,当元素被删除后,可将它所占用的空间放入到free-list中。

    \item{\textbf{Solution}}\\
	\item{Equally Divided} : \fbox{时间复杂度O($n$) , 空间复杂度O($n$) }\\
		\begin{lstlisting}
		int stackSize = 300;
		int[] buffer = new int [stackSize * 3];
		int[] stackPointer = {0, 0, 0}; // stack pointers to track top elem
		void push(int stackNum, int value) {
			/* Find the index of the top element in the array + 1, and increment the stack pointer */
			int index = stackNum * stackSize + stackPointer[stackNum] + 1;
			stackPointer[stackNum]++;
			buffer[index] = value;	
		}
		int pop(int stackNum) {
			int index = stackNum * stackSize + stackPointer[stackNum];
			stackPointer[stackNum]--;
			int value = buffer[index];
			buffer[index]=0;
			return value;
		}
		int peek(int stackNum) {
			int index = stackNum * stackSize + stackPointer[stackNum];
			return buffer[index];
		}
		boolean isEmpty(int stackNum) {
			return stackPointer[stackNum] == stackNum*stackSize;
		}
		\end{lstlisting}
\end{description}

