    
\begin{description}
    \item{\textbf{Question}}:\\%textbf加粗
		All DNA is composed of a series of nucleotides abbreviated as A, C, G, and T, for example: "ACGAATTCCG". When studying DNA, it is sometimes useful to identify repeated sequences within the DNA.\\
		Write a function to find all the 10-letter-long sequences (substrings) that occur more than once in a DNA molecule.\\

    \item{\textbf{Example}}\\
		Given s = "AAAAACCCCCAAAAACCCCCCAAAAAGGGTTT".\\
		Return: ("AAAAACCCCC", "CCCCCAAAAA").\\

    \item{\textbf{Anslysis}}\\
		查找子串的存在性可以用hash来加速,这里的问题是如何减少hash中存储的对象的大小?\\
		因为子串全部为10的长度,如果直接存储的话,每个子串都消耗10字节。因为子串中只会出现四种元素,所以可以考虑将子串转换为数字后处理。\\
		此外,需要注意在生成下一个子串的哈希值时可以参考上一个子串的哈希值。\\
		A is 0x41, C is 0x43, G is 0x47, T is 0x54\\
		A is 0101, C is 0103, G is 0107, T is 0124\\

    \item{\textbf{Solution}}\\
	\item{Hash} : \fbox{时间复杂度O($n$) , 空间复杂度O(1) }\\
		\begin{lstlisting}
vector<string> findRepeatedDnaSequences(string s) {
    unordered_map<int, int> m;
    vector<string> r;
    int t = 0, i = 0, ss = s.size();
    while (i < 9)
        t = t << 3 | s[i++] & 7;
    while (i < ss)
        if (m[t = t << 3 & 0x3FFFFFFF | s[i++] & 7]++ == 1)
            r.push_back(s.substr(i - 10, 10));
    return r;
}		\end{lstlisting}

\end{description}

