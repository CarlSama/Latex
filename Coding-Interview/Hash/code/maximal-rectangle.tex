    
\begin{description}
    \item{\textbf{Question}}:\\%textbf加粗
		Given an array of n positive integers and a positive integer s, find the minimal length of a subarray of which the sum ≥ s. If there isn't one, return 0 instead.

    \item{\textbf{Example}}\\
		Given the array [2,3,1,2,4,3] and s = 7, the subarray [4,3] has the minimal length under the problem constraint.

    \item{\textbf{Anslysis}}\\
		类似于minimal-window-substring,维护双指针来表征区间,然后不停的更新区间即可。因为每个元素最多被访问2次,所以时间复杂度为O(n)。\\
		因为二分只适用于已排序的对象,所以我们必须获得一个已排序的对象。直接对nums做sort显然没有意义。考虑到nums中都是positive的,所以可以构造出累加和来作为待处理的对象。这样就满足了二分的基础。\\

    \item{\textbf{Solution}}\\
	\item{Hash} : \fbox{时间复杂度O($n$) , 空间复杂度O($n$) }\\
		\begin{lstlisting}
class Solution {
	public:
		int countPrimes(int n) {
			vector<bool> isPrime(n+1,true);

			int end = sqrt(n);
			for(int i=2;i<=end;++i) {
				if(isPrime[i]) {
					for(int j = i * i ; j< n; j += i) 
						isPrime[j] = false;
				}
			}

			int count = 0;
			for(int i=2;i<n;++i) {
				if(isPrime[i])
					++count;
			}

			return count;
		}
};
		\end{lstlisting}

	\item{Binary Search} : \fbox{时间复杂度O($nlgn$) , 空间复杂度O($n$) }\\
		\begin{lstlisting}
    private int solveNLogN(int s, int[] nums) {
        int[] sums = new int[nums.length + 1];
        for (int i = 1; i < sums.length; i++) sums[i] = sums[i - 1] + nums[i - 1];
        int minLen = Integer.MAX_VALUE;
        for (int i = 0; i < sums.length; i++) {
            int end = binarySearch(i + 1, sums.length - 1, sums[i] + s, sums);
            if (end == sums.length) break;
            if (end - i < minLen) minLen = end - i;
        }
        return minLen == Integer.MAX_VALUE ? 0 : minLen;
    }

    private int binarySearch(int lo, int hi, int key, int[] sums) {
        while (lo <= hi) {
           int mid = (lo + hi) / 2;
           if (sums[mid] >= key){
               hi = mid - 1;
           } else {
               lo = mid + 1;
           }
        }
        return lo;
    }

		\end{lstlisting}
\end{description}

