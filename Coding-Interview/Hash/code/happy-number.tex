    
\begin{description}
    \item{\textbf{Question}}:\\%textbf加粗
		Write an algorithm to determine if a number is "happy".\\
		A happy number is a number defined by the following process: Starting with any positive integer, replace the number by the sum of the squares of its digits, and repeat the process until the number equals 1 (where it will stay), or it loops endlessly in a cycle which does not include 1. Those numbers for which this process ends in 1 are happy numbers.

    \item{\textbf{Example}}\\
		 19 is a happy number\\
		 12 + 92 = 82\\
		 82 + 22 = 68\\
		 62 + 82 = 100\\
		 12 + 02 + 02 = 1

    \item{\textbf{Anslysis}}\\
		需要查找当前处理的结果在之前是否出现过。\\

    \item{\textbf{Solution}}\\
	\item{Hash} : \fbox{时间复杂度O($n$) , 空间复杂度O($n$) }\\
		\begin{lstlisting}
class Solution {
	public:
		int getSquareSum(int n) {
			int squareSum = 0;
			while(n) {
				squareSum += ((n%10) * (n%10));
				n /= 10;
			}
			return squareSum;
		}
		bool isHappy(int n) {
			int squareSum = getSquareSum(n);
			int beginLoop = squareSum;
			while(squareSum != 1) {
				squareSum = getSquareSum(squareSum);

				if(squareSum == beginLoop)
					return false;
			}
			return true;
		}
};

		\end{lstlisting}

\end{description}

